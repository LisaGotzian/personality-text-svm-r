\section{Introduction}
\label{intro}
One of the most important concepts in psychology is an individual's personality. As it is stable over the course of 20 years \cite{mccrae_introduction_1992}, its effect on all kinds of different behaviors and concepts such as job performance \cite{tett_personality_2006}, happiness \cite{ozer_personality_2006} and marital satisfaction \cite{kelly_personality_1987} has been evaluated. The insights of personality research are valuable for any research conducted around humans such as organizational psychology, human-machine interaction or customer management just to name a few.\\
The most famous model to describe personality is the \textbf{Five Factor Model}, the Big 5 \cite{mccrae_introduction_1992}. The model states how each personality can be described using 5 bipolar dimensions: \emph{openness to new experience, conscientiousness, extraversion, agreeableness} and \emph{neuroticism}.\\
The model has been cross-validated and is broadly accepted. It is usually measured using tests like the NEO PI-R \cite{mccrae_introduction_1992}. These 60 min long tests are time-consuming and require lots of efforts when recruiting participants. In addition, especially personality data is strongly protected as this shares very private information \cite{kosinski_facebook_2015}. Thus, datasets with information on the Big 5 are hard to accumulate and are potentially dangerous to work with.\\
In contrast to the Big 5, the \textbf{Myers-Briggs Type Indicator} evaluates personality on 4 dimensions: \emph{Introversion/Extraversion} (how one gains energy), \emph{Sensing/iNtuition} (how one processes information), \emph{Thinking/Feeling} (how one makes decisions), and \emph{Judging/Perceiving} (how one presents herself or himself to the outside world) \cite{myers_mbti_1998}. The combination of all 4 dimensions yields 16 different personality types based on the works by Carl Jung \cite{jung_personality_2014}. Though these dimensions correlate with 4 of the 5 dimensions of the Big 5 \cite{mccrae_reinterpreting_1989}, it lacks the last dimension, \emph{neuroticism}. The MBTI however is not as broadly accepted in the scientific community as it fails to provide a valid model with clearly distinguishable dimensions \cite{boyle_myers-briggs_1995,mccrae_reinterpreting_1989}.\\
For research, the MBTI provides use nevertheless: first, especially in the corporate world, the MBTI enjoys great popularity. It is much easier to retrieve correctly labeled data, e.g. from Twitter  \cite{plank_personality_2015} or from reddit \cite{gjurkovic_reddit:_2018}. In both cases, users deliberately show their personality type and contribute to datasets with several thousand entries.\\
Second, due to its closeness to the Big 5, the MBTI might be able to predict certain Big 5 features, in return leading to usable results but with the advantage of more accessible data. Datasets on personality especially involving the Big 5 are hard to get ahold of due to privacy issues or maintenance reasons \cite{stillwell_mypersonality_2018}.\\
Recently, machine learning researchers have started to infer the Big 5 from more indirect data. Instead of elaborate tests, they predicted personality based on text \cite{mairesse_using_2007,pennebaker_linguistic_1999}
like Twitter posts \cite{golbeck_predicting_2011,quercia_our_2011}, other social media posts \cite{schwartz_personality_2013} and emails \cite{oberlander_language_2006}. Personality can also be inferred from general social media behavior as likes, interactions and other digital records \cite{kosinski_private_2013,kosinski_manifestations_2014,youyou_computer-based_2015}. The results come with an accuracy of 0.4-0.5 and up to 0.8 for individual dimensions, beating even the ratings by close family members \cite{youyou_computer-based_2015}.\\
This paper intends to (1) predict MBTI types based on social network posts as done with the Big 5 to then (2) determine if the MBTI is a valid measure for text data. Based on this, one could derive the user's Big 5 which would pose a similar threat on privacy as the Big 5 itself.