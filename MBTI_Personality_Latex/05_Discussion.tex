\section{Discussion}
\label{sec:discussion}
This paper intended to (1) predict MBTI types based on social network posts as done with the Big 5 to then (2) determine if the MBTI is a valid measure for text data. The results show that text can indeed be a good predictor for people's MBTI personality types.
The analysis also showed that similar personality types tend to share communities and hang out with each other. This network effect can be used for future research and implies that network analysis of social media profiles might also predict personality types well.\\
Though the MBTI is not as accepted as the Big 5 Model, it enjoys a high social validity and a broad acceptance in the population. As many people know their MBTI type and are willing to share it online, they grant access to their personal data including their personality type and allow for easily collectible datasets. With the results of this MBTI analysis, one could then derive 4 of the 5 Big Five dimensions as they correlate, giving estimates for a more valid model and easing out the disadvantage of the MBTI.\\
Even if researchers have an interest in these large datasets, the implication that personality can be predicted based on public data such as social media posts is a major threat to a user's data privacy. If the MBTI manages to predict the Big 5, the call for a privacy threat is equally valid for the MBTI. If then the use of data on the Big 5 is heavily restricted \cite{kosinski_facebook_2015}, one could consider doing the same for MBTI data.\\
Such private data has been abused in the past, especially concerning targeted ads during the US-American elections by Cambridge Analytica\cite{krogerus_ich_2018}\footnote{``Cambridge Analytica - The Power of Big Data and Psychographics", https://www.youtube.com/watch?v=n8Dd5aVXLCc, retrieved March 14, 2019.}. The specific manipulation of user's voting preferences raised the question if targeted ads and thus predicting personality is ethical in the first place.\\
Participants seem to be willing to share their data with researchers, e.g. in the most prominent example over 30\% of myPersonality participants voluntarily shared the contents of their Facebook profiles, together with their personality, intelligence, and other psychometric scores \cite{stillwell_mypersonality_2018}. However, one can hardly speak of \textit{informed consent}\cite{fleming_telehealth_2009}, especially because the users will not be able to estimate what could be done with the data. This is why Kosinski et al. propose a consent form with clear intentions on how the data will be used \cite{kosinski_facebook_2015}. This might not solve the overall issue but could be part of a solution to this ethical problem.

\paragraph{Limitations} Not all dimensions delivered equally valid results. Due to the lack of data for the intuition/sensing dimension, a reliable estimate for the correlated agreeableness dimension in the Big 5 cannot be provided using this dataset.\\
Though previous researchers reported lower performances around 0.4 or 0.5 \cite{gjurkovic_reddit:_2018,kosinski_private_2013,kosinski_manifestations_2014} and this model seems to perform fairly well, the fact that the dataset comes from a personality forum cannot be discarded. The personality enthusiasts in that forum will probably be much more aware of their type and identify themselves more with it. This would in return make the type much better identifiable for an algorithm in their expressions and their posts. Thus, the results in this analysis will probably be higher than in a more randomly sampled dataset. In future research, a less specific dataset can shed light on the reliability of predicting the Big 5 using the MBTI and thus on the following privacy issues.